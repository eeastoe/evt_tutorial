\documentclass[dvipsnames]{beamer}
%comes before \usetheme else the slide titles are cut off...
\useoutertheme{infolines}
\usetheme[height=7mm]{Boadilla}
%\usecolortheme[RGB={70, 130, 138}]{structure}
\usecolortheme[RGB={255, 140, 0}]{structure}
\usepackage{psfrag}
\usepackage{graphicx}
\usepackage{multirow}
\usepackage[hang]{subfigure}
\usepackage{amsmath,mathtools}
\usepackage{verbatim}
\setbeamertemplate{itemize items}[ball]
\setbeamertemplate{navigation symbols}{}

\newcommand{\blue}{\textcolor{CornflowerBlue}}
\newcommand{\green}{\textcolor{Green}}


\renewcommand{\subfigcapskip}{8pt}
\author[Emma Eastoe]{Emma Eastoe}
\institute[Lancaster University]{Department of Mathematics and
  Statistics \\  Lancaster University, UK}
\title[Extremes@Lancaster]{Extreme Value Analysis}
\date

\titlegraphic{
%\includegraphics[scale=0.3]{/home/eastoee/Research/AmberLeeson/MEMOG/Exeter/Plots/LUmathstatslogo.pdf}\hspace{2.75cm}
\includegraphics[scale=0.3]{/home/eastoee/Research/AmberLeeson/MEMOG/Exeter/Plots/LUmathstatslogo.pdf}%\hspace{2.75cm}
}

\begin{document}

\begin{frame}[plain,noframenumbering]
\titlepage
\end{frame}


\begin{frame}[plain,noframenumbering]
\tableofcontents
\end{frame}


\section{Introduction}
\begin{frame}
\frametitle{Introduction}

\begin{itemize}
\item
Statistical models aimed at capturing behaviour of very largest, or very smallest, observations in a data set.
\item
Many applications in the environmental sciences: air pollution, hydrology, temperatures, wind speed, precipitation, wave heights,...
\item
Goal: estimate return levels or the probability of an unusually large (small) event.
\end{itemize}

\end{frame}


\begin{frame}


\begin{block}{Return levels}
\begin{itemize}
\item
$N$-year return level, the value a process is expected to exceed, once every $N$ years. Where $N$ could be 50, 100, 500, ...
\item
If you are unlucky, this return level {\it could} be exceeded in two successive years even if $N$ is large.
\end{itemize}
\end{block}

\begin{block}{Probability of a large event}
To estimate the probability that a process exceeds $x$:
\begin{itemize}
\item
If $x$ is `moderate' do this empirically: calculate the proportion of observations (data points) above $x$;
\item
If $x$ is high may have no observations above $x$ so empirical method doesn't work: use a model and extrapolate.
\end{itemize}
\end{block}

\end{frame}

\begin{frame}
%\subsection{Greenland temperatures}
\frametitle{Greenland temperatures}

\begin{columns}[T]
\begin{column}{0.5\textwidth}
\vfill
\begin{itemize}
\item
Greenland is mostly ice sheet
\item
Why worry about extremely high temperatures?
\item
Rising temperatures $\rightarrow$ \\
more positive degree days $\rightarrow$ \\greater melting $\rightarrow$ \\increased global sea level
\end{itemize}
\vfill
\end{column}
\hfill
\begin{column}{0.5\textwidth}
Daily max temperatures, Kangerlussuaq (1975--2015)
\begin{figure}
\centering
%{\includegraphics[scale=0.2]{/home/eastoee/Research/AmberLeeson/MEMOG/Exeter/Plots/topographic-map-of-greenland-max.jpg}}
%{\includegraphics[scale=0.2]{/home/eastoee/Research/AmberLeeson/MEMOG/Exeter/Plots/GC-net-stations.pdf}}
{\includegraphics[scale=0.3]{/home/eastoee/ShortCourses/GlaciologyExtremes/Plots/kanger_ts.pdf}}
\end{figure}
\end{column}
\end{columns}
\end{frame}

\section{Maxima and minima}

\begin{frame}
\frametitle{Annual maxima and minima}

The simplest way to characterise an extreme event is to look at the annual maxima/minima. We may ask:


\begin{itemize}
\item
What is the largest maxima (smallest minima) that we might expect to see
\begin{itemize}
\item
In 10-years?
\item
In 100-years?
\item
Ever?
\end{itemize}
\item
What is the chance that the annual maximum exceeds a certain high and previously unobserved value?
\item
Is the behaviour of the annual minima/maxima changing over time?
\end{itemize}
\end{frame}

\begin{frame}
\frametitle{Kangerlussuaq}

\begin{columns}[T]
\begin{column}{0.5\textwidth}
\vfill
Annual maxima
\begin{figure}
\centering
{\includegraphics[scale=0.3]{/home/eastoee/ShortCourses/GlaciologyExtremes/Plots/kangerAnnMax.pdf}}
\end{figure}
\vfill
\end{column}
\hfill
\begin{column}{0.5\textwidth}
Annual minima
\begin{figure}
\centering
{\includegraphics[scale=0.3]{/home/eastoee/ShortCourses/GlaciologyExtremes/Plots/kangerAnnMin.pdf}}
\end{figure}
\end{column}
\end{columns}
\end{frame}

\subsection{The generalised extreme value (GEV) distribution}
\begin{frame}
\frametitle{Models for maxima and minima}
\begin{itemize}
\item
Generalised extreme value distribution
\item
Defined by it's cumulative distribution function. Let $X$ represent an annual maximum (minimum) then
\begin{eqnarray*}
G(x)=\Pr[X\leq x]=\exp\left\{-\left[1+\xi\left(\frac{x-\mu}{\sigma}\right)\right]^{-1/\xi}\right\}
\end{eqnarray*}
\item
Three parameters (unknowns): location $\mu$, scale $\sigma\in(0,\infty)$ and shape $\xi$
\item
Shape determines how fast the distribution decays i.e. how quickly the quantiles get large. 
\end{itemize}

\end{frame}

\begin{frame}
\frametitle{GEV$(0,1,\xi)$}
Shapes: $\xi=-0.2$ (turquoise), $\xi=0$ (purple) and $\xi=0.2$ (orange)
\begin{columns}[T]
\begin{column}{0.5\textwidth}
\vfill
Densities 
\begin{figure}
\centering
{\includegraphics[scale=0.3]{/home/eastoee/ShortCourses/GlaciologyExtremes/Plots/gev_density.pdf}}
\end{figure}
\vfill
\end{column}
\hfill
\begin{column}{0.5\textwidth}
Upper quantiles
\begin{figure}
\centering
{\includegraphics[scale=0.3]{/home/eastoee/ShortCourses/GlaciologyExtremes/Plots/gev_quantile.pdf}}
\end{figure}
\end{column}
\end{columns}
\end{frame}

\begin{frame}
\frametitle{GEV as a statistical model}
\begin{itemize}
\item
Useful to model any data set which is contains maxima or minima
\item
Condition: maxima/minima have been taken over a `large enough' number of underlying observations
\begin{itemize}
\item
Fine to use for annual max/min of daily observations
\item
Not so good to use for daily max/min of hourly observations 
\end{itemize}
\item
Estimation of parameters via any method of statistical inference
\begin{itemize}
\item 
Maximum likelihood, Bayes, L-moments
\item 
All of these implemented in \textsf{R} package \texttt{extRemes}
\end{itemize}
\end{itemize}
\end{frame}

\begin{frame}
\frametitle{Return levels}
\begin{itemize}
\item
For the GEV these are directly related to quantiles.
\item
Let $x_N$ be the $N$-year return level, then to find $x_N$ assuming a GEV model, solve
\begin{eqnarray*}
\Pr[X\leq x_N]=\exp\left\{-\left[1+\xi\left(\frac{x-\mu}{\sigma}\right)\right]^{-1/\xi}\right\}=1-\frac{1}{N}
\end{eqnarray*}
\item
Gives
\begin{eqnarray*}
x_N=\frac{1}{\xi}\left\{\left[-\log(1-1/N)\right]^{-\xi}-1\right\}
\end{eqnarray*}
\item
Again implemented in \texttt{extRemes}
\end{itemize}
\end{frame}

\begin{frame}[fragile]
\frametitle{Implementation in \textsf{R}}
\begin{itemize}
\item
Quite a few packages including \texttt{extRemes}, \texttt{texmex}, \texttt{evd} and \texttt{ismev}
\item
For this course we will use \texttt{extRemes}
\item
Has a single function for fitting the various EVA models.
\end{itemize}
\end{frame}

\begin{frame}[fragile]
\frametitle{Kangerlussuaq again}
\begin{itemize}
\item
Load the data into \textsf{R},
\begin{verbatim}
> annMaxKanger <- read.csv("kangerMax.csv")
\end{verbatim}
Data frame with two columns: \texttt{Year} and \texttt{Max}
\item
Fit the model
\begin{verbatim}
> max.fit.1 <- fevd(x=Max,data=annMaxKanger)
\end{verbatim}
\end{itemize}
\end{frame}

\begin{frame}[fragile]
\begin{itemize}
\item
To obtain parameter estimates, standard errors etc 
\begin{verbatim}
> summary(max.fit.1)
\end{verbatim}
\item
Important parts of the output
\begin{verbatim}
 Estimated parameters:
  location      scale      shape 
21.4478281  1.1863160 -0.1135755 

 Standard Error Estimates:
 location     scale     shape 
0.2095874 0.1471525 0.1088952 

 AIC = 140.7024 
\end{verbatim}
\end{itemize}
\end{frame}

\begin{frame}[fragile]
Visual diagnostics of model fit:
\begin{verbatim}
> plot(max.fit.1)
\end{verbatim}
\begin{figure}
\centering
{\includegraphics[scale=0.3]{/home/eastoee/ShortCourses/GlaciologyExtremes/Plots/kanger_gev_plot.pdf}}
\end{figure}
\end{frame}

\begin{frame}[fragile]
Estimate return levels using
\begin{verbatim}
> rl <- return.level(max.fit.1,return.period=seq(5,500,by=5))
\end{verbatim}
and to plot
\begin{verbatim}
> plot(max.fit.1,type="rl")
\end{verbatim}

%\begin{columns}[T]
%\begin{column}{0.5\textwidth}
%Return period on original scale
%\vfill
%\begin{figure}
%\centering
%{\includegraphics[scale=0.3]{/home/eastoee/ShortCourses/GlaciologyExtremes/Plots/kanger_gev_rl.pdf}}
%\end{figure}
%\end{column}
%\begin{column}{0.5\textwidth}
%Return period on log scale
%\vfill
\begin{figure}
\centering
{\includegraphics[scale=0.3]{/home/eastoee/ShortCourses/GlaciologyExtremes/Plots/kanger_gev_rl.pdf}}
\end{figure}
%\end{column}
%\end{columns}
\end{frame}

\subsection{Regression-type models}

\begin{frame}
\frametitle{Regression modelling}
Model discussed above assumes data are stationary over time. It is more likely that either (or both):
\begin{itemize}
\item
There is a trend in the maxima over time (climate change?)
\item
One or more physical variables could be used to help explain changes in the maxima
\end{itemize}
Model this using regression-type techniques.
\end{frame}

\begin{frame}
\begin{itemize}
\item
Write location and/or scale parameter as linear functions of covariate(s) e.g.
\begin{eqnarray*}
\mu(\mathrm{time})&=&\mu_0+\mu_1\mathrm{time}\\
\log\sigma(\mathrm{time})&=&\sigma_0+\sigma_1\mathrm{time}
\end{eqnarray*}
\item
`log-link' for scale to ensure $\sigma(\mathrm{time})>0$ for all time
\item
In general, keep shape $\xi$ constant
\item
More unknown parameters to estimate!
\item
Compare models using likelihood ratio test or AIC to only include significant covariate(s)
\end{itemize}
\end{frame}

\begin{frame}[fragile]
\frametitle{Regression models in \textsf{R}}
\begin{verbatim}
> annMaxKanger$Time <- annMaxKanger$Year-1974
> max.fit.2 <- 
      fevd(x=Max,data=annMaxKanger,location.fun=~1+Time)
> summary(max.fit.2)
\end{verbatim}

The fitted model is 
\begin{eqnarray*}
\mu(\mathrm{year})&=&20.4+0.049\times(\mathrm{year}-1974)\\
\sigma&=&0.99\\
\xi&=&0.0056
\end{eqnarray*}
The standard error for the `year' coefficient is 0.015.
\end{frame}

\begin{frame}[fragile]
\frametitle{Model selection}
\begin{itemize}
\item
If models are nested, use the likelihood ratio test
\item
Enter the simpler model first, in this case \texttt{max.fit.1}
\end{itemize}
Using the \texttt{extRemes} package:

\begin{verbatim}
> lr.test(max.fit.1,max.fit.2)
\end{verbatim}

The output looks like this:
\begin{verbatim}
Likelihood-ratio Test

data:  MaxMax
Likelihood-ratio = 8.5274, chi-square critical value = 3.8415, alpha =
0.0500, Degrees of Freedom = 1.0000, p-value = 0.003498
alternative hypothesis: greater
\end{verbatim}
$p=0.0035<0.05$ therefore there is evidence of a time trend in the location parameter
\end{frame}

\begin{frame}[fragile]
\frametitle{Effective return levels}
These are $N$-year return levels for each value of the covariate 
\begin{verbatim}
> plot(max.fit.2,type="rl")
\end{verbatim}
\begin{figure}
\centering
{\includegraphics[scale=0.3]{/home/eastoee/ShortCourses/GlaciologyExtremes/Plots/kanger_effective_rl.pdf}}
\end{figure}
\end{frame}

%%%%%%%%%%%%%%%%%%%%%%%%%%%%%%%%%%%%%%%%%%%%%%%%%%%%%%%%%%%%%%%%%%%%%%%%%%%%%%%%%%%%%%%%%%%%%%%%%%%%%%%%%%%%%%%%%%%%%%%%%%%%%%%%%%%%%%%%%%%%%%%%%%%%%%%%%%%%%%%%

\section{Peaks over Threshold}

\begin{frame}
\frametitle{Peaks over Threshold (PoT)}
\begin{itemize}
\item
Alternative to modelling only maxima/minima
\item
Allows us to model all unusually large (or small) events
\item
Requires identification of these events
\item
But a more efficient use of the data than just taking maxima/minima
\end{itemize}
\end{frame}

\begin{frame}
\frametitle{Overview of PoT approach}
\begin{itemize}
\item
Choose a high threshold: any observation above this is classified as an extreme event
\item
Model
\begin{itemize}
\item
Rate - how often do they occur?
\item
Size - how large are they?
\end{itemize}
of threshold exceedances.
\end{itemize}
Note on threshold identification: visual aids exist to help with this e.g. mean residual life plot.
\end{frame}

\subsection{The generalised Pareto (GP) distribution}

\begin{frame}
\frametitle{Generalised Pareto distribution}
\begin{itemize}
\item
The generalised Pareto (GP) distribution is used to model the size of threshold exceedances
\item
The full cumulative distribution function for an exceedance is given by 
\begin{eqnarray*}
\Pr[X\leq x]=1-\phi\left[1+\xi\left(\frac{x-u}{\psi}\right)\right]^{-1/\xi}
\end{eqnarray*}
\item
Three unknown parameters: rate $\phi\in [0,1]$, scale $\psi\in(0,\infty)$ and shape $\xi$
\item
As with GEV, parameter estimation by likelihood, Bayesian, L-moments,...
\item
Relies on threshold being very high
\end{itemize}
\end{frame}

\begin{frame}
\frametitle{Kangerlussuaq: daily maxima temperatures}
Choose the 90\% quantile as a threshold (purple line)
\begin{figure}
\centering
{\includegraphics[scale=0.3]{/home/eastoee/ShortCourses/GlaciologyExtremes/Plots/kanger_threshold_zoom.pdf}}
\end{figure}
\end{frame}

\begin{frame}[fragile]
\frametitle{Implementation of GP in \textsf{R}}
Can use the same \texttt{fevd} function to fit the GP model as we used for the GEV model:
\begin{verbatim}
> kanger <- read.csv("kangerTemp.csv")
> thresh <- quantile(kanger$Temp,0.9) #define threshold
> gp.fit.1 <- fevd(x=Temperature,data=kanger,threshold=thresh,type="GP")
> summary(gp.fit.1)
\end{verbatim}
From the last command we see that $\hat\psi=2.28$ and $\hat\xi=-0.27$. To find $\hat\phi$,
\begin{verbatim}
> gp.fit.1$rate
\end{verbatim}
and $\hat\phi=0.098$ (why is this not surprising?!)
\end{frame}

\begin{frame}[fragile]
As for the GEV model, use visual diagnostics to check model fit,
\begin{verbatim}
> plot(gp.fit.1)
\end{verbatim}
\begin{figure}
\centering
{\includegraphics[scale=0.3]{/home/eastoee/ShortCourses/GlaciologyExtremes/Plots/kanger_gp_diag.pdf}}
\end{figure}
\end{frame}

\begin{frame}[fragile]
And return levels for the daily temperature (not the annual maxima now):
\begin{verbatim}
> plot(gp.fit.1,type="rl")
\end{verbatim}
\begin{figure}
\centering
{\includegraphics[scale=0.3]{/home/eastoee/ShortCourses/GlaciologyExtremes/Plots/kanger_gp_rl.pdf}}
\end{figure}
\end{frame}

%%%%%%%%%%%%%%%%%%%%%%%%%%%%%%%%%%%%%%%%%%%%%%%%%%%%%%%%%%%%%%%%%%%%%%%%%%%%%%%%%%%%%%%%%%%%%%%%%%%%%%%%%%%%%%%%%%%%%%%%%%%%%%%%%%%%%%%%%%%%%%%%%%%%%%%%%%%%%%%%%%%%%%%%%%%%%%%%%%%%%%%%%%%%%%%%%%%%%%%%%%%%%%%%%%%%%%%%%%%%%%%%%%%%%%%%%%%%%%%

\subsection{Cluster identification}

\begin{frame}
\frametitle{Modelling event maxima only}
\begin{itemize}
\item
The diagnostics and return level plot suggest that the model could do better at describing the highest temperatures
\item
Model could be too simplistic
\item
Perhaps behaviour changes over time
\item
Or extremes are not independent
\end{itemize}
\end{frame}

\begin{frame}
\frametitle{Event identification}
\begin{itemize}
\item
So far have thought of each threshold exceedance as a separate event
\item
What if exceedances occur in groups (clusters)?
\item
Can model cluster {\it maxima} using the GP model
\item
Need a way to identify clusters: number of algorithms
\item
Use the {\it runs method}: Exceedances separated by 
\begin{itemize}
\item
fewer than $r$ consecutive non-exceedances belong to same cluster;
\item
more than $r$ consecutive non-exceedances belong to independent clusters.
\end{itemize}
\end{itemize}
\end{frame}

\begin{frame}[fragile]
\frametitle{Cluster Identification in \textsf{R}}
Use the \texttt{decluster} function:
\begin{verbatim}
> kangerDecl <- decluster(kanger$Temperature,threshold=thresh
     ,method="runs",r=3)
\end{verbatim}
To get a summary of the declustering
\begin{verbatim}
> print(kangerDecl)
\end{verbatim}

\begin{itemize}
\item
Shows 223 clusters. 
\item
Should assess sensitivity to choice of different run lengths.
\end{itemize}
\end{frame}

\begin{frame}[fragile]
\frametitle{Extremal Index}
\subsection{Extremal Index}
\begin{block}{Extremal Index}
Often denoted $\theta$ is a measure of the strength of extremal dependence. 
\begin{itemize}
\item
Lies in the interval $[0,1]$. 
\item
Stronger dependence as $\theta$ gets closer to 0.
\item
Mean cluster size is reciprocal of extremal index.
\item
Independent series have $\theta=1$ but $\theta=1$ does not imply that the series is independent, merely that the threshold exceedances are independent. 
\end{itemize}
\end{block}
\end{frame}

\begin{frame}[fragile]
\begin{itemize}
\item
The extremal index can be estimated using the runs or intervals method. 
\item
\texttt{decluster} and \texttt{decluster.runs} in \texttt{extRemes} automatically output the intervals estimate even if the runs method is specified. 
\item
Instead use \texttt{extremalindex} function:
\begin{verbatim}
> extremalindex(kanger$Temperature,threshold=thresh
     ,method="runs",run.length=3)
\end{verbatim}
\item
Gives $\hat\theta=0.158$, with a mean cluster size of $1/0.158 =6.3$.
\end{itemize}
\end{frame}

\begin{frame}[fragile]
Can look at the clusters using 
\begin{verbatim}
> plot(kangerDecl)
\end{verbatim}
\begin{figure}
\centering
{\includegraphics[scale=0.3]{/home/eastoee/ShortCourses/GlaciologyExtremes/Plots/kanger_decl.pdf}}
\end{figure}
\end{frame}

\begin{frame}[fragile]
Refit GP to cluster maxima only, and check diagnostics.
\begin{verbatim}
> gp.fit.2 <- fevd(x=kangerDecl,threshold=thresh,type="GP")
> plot(gp.fit.2)
\end{verbatim}
For cluster max only $\hat\psi=0.24$ and $\hat\xi=0.040$. Diagnostics much better!
\begin{figure}
\centering
{\includegraphics[scale=0.3]{/home/eastoee/ShortCourses/GlaciologyExtremes/Plots/kanger_decl_gp.pdf}}
\end{figure}
\end{frame}

%%%%%%%%%%%%%%%%%%%%%%%%%%%%%%%%%%%%%%%%%%%%%%%%%%%%%%%%%%%%%%%%%%%%%%%%%%%%%%%%%%%%%%%%%%%%%%%%%%%%%%%%%%%%%%%%%%%%%%%%%%%%%%%%%%%%%%%%%%%%%%%%%%%%%%%%%%%%%%%%%%%%%%%%%%%%%%%%%%%%%%%%%%%%%%%%%%%%%%%%%%%%%%%%%%%%%%%%%%%%%%%%%%%%%%%%%%%%%

\section{Summary}

\begin{frame}
\frametitle{Summary}
\begin{itemize}
\item
Two main EVA models: Generalised Extreme Value (GEV) and generalised Pareto (GP) distributions
\item
GEV appropriate for annual maxima or minima data
\item
GP used to model threshold exceedances in PoT approach
\item
Can extend both models to include regression terms
\item
For PoT approach might want to check for clustering and model only cluster maxima
\item
Return levels can be easily produced for both GEV and GP models; if regression terms are included then effective return levels will be produced
\end{itemize}
\end{frame}

\subsection{Further topics}

\begin{frame}
\frametitle{Further areas of interest}
\begin{itemize}
\item
Including regression terms in the GP model
\item
Statistical downscaling of extremes
\item
Spatial modelling of extreme events; Gaussian process based models not necessarily appropriate
\item
Multivariate modelling of extremes: joint extremal behaviour of e.g. two variables at a single location, the same variable at two locations,...
\end{itemize}
\end{frame}

\end{document}
